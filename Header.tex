% Typography
\usepackage{fontspec}
\defaultfontfeatures{Ligatures=TeX}
\setmainfont{Source Serif Pro}
\setmonofont[Scale=MatchLowercase]{Source Code Pro}
\usepackage{microtype}
\usepackage{setspace}
\usepackage{csquotes}

% Colors, links, syntax highlighting
\usepackage[dvipsnames,table]{xcolor}
\usepackage[pagebackref]{hyperref}
\hypersetup{
    pdftitle={Crypto 101},
    pdfauthor={Laurens Van Houtven (lvh)},
    colorlinks,
    linkcolor=Sepia,
    citecolor=Periwinkle
}
\usepackage{minted}
\usepackage[acronym,toc]{glossaries}
\newacronym{AEAD}{AEAD}{Authenticated Encryption with Associated Data}
\newacronym{AES}{AES}{Advanced Encryption Standard}
\newacronym{BEAST}{BEAST}{Browser Exploit Against SSL/TLS}
\newacronym{CBC}{CBC}{Cipher Block Chaining}
\newacronym{DES}{DES}{Data Encryption Standard}
\newacronym{FIPS}{FIPS}{Federal Information Processing Standards}
\newacronym{GCM}{GCM}{Galois Counter Mode}
\newacronym{HKDF}{HKDF}{HMAC-based (Extract-and-Expand) Key Derivation Function}
\newacronym{HSTS}{HSTS}{HTTP Strict Transport Security}
\newacronym{IV}{IV}{\gls{initialization vector}}
\newacronym{KDF}{KDF}{key derivation function}
\newacronym{MAC}{MAC}{message authentication code}
\newacronym{OCB}{OCB}{offset codebook}
\newacronym{OTR}{OTR}{off-the-record}
\newacronym{PRF}{PRF}{pseudorandom function}
\newacronym{PRP}{PRP}{pseudorandom permutation}

\newglossaryentry{secret-key encryption}{
  name=secret-key encryption, description={Encryption that uses the
  same key for both encryption and decryption. Also known as
  symmetric-key encryption. Contrast with \gls{public-key encryption}}
  }
\newglossaryentry{symmetric-key encryption}{
  name=symmetric-key encryption, description={See \gls{secret-key
  encryption}}}
\newglossaryentry{keyspace}{
  name=keyspace, description={The set of all possible keys}}
\newglossaryentry{block cipher}{
  name=block cipher, description={Symmetric encryption algorithm that
  encrypts and decrypts blocks of fixed size}, }
\newglossaryentry{stream cipher}{
  name=stream cipher, description={Symmetric encryption algorithm that
  encrypts streams of arbitrary size} }
\newglossaryentry{mode of operation}{
  name=mode of operation, description={Generic construction that
  encrypts and decrypts streams, built from a block
  cipher},plural=modes of operation }
\newglossaryentry{Feistel network}{
  name=Feistel network, description={Generic construction for a \gls{block
  cipher} that gives decryption by means of simply reversing the order of
  rounds (and steps), but not the functions themselves.} }
\newglossaryentry{ECB mode}{
  name=ECB mode, description={Electronic code book mode; mode of
  operation where plaintext is separated into blocks that are
  encrypted separately under the same key. The default mode in many
  cryptographic libraries, despite many security issues} }
\newglossaryentry{CBC mode}{
  name=CBC mode, description={Cipher block chaining mode; common mode
  of operation where the previous ciphertext block is XORed with the
  plaintext block during encryption. Takes an initialization vector,
  which assumes the role of the \enquote{block before the first
  block}}}
\newglossaryentry{initialization vector}{
  name=initialization vector, description={Data used to initialize
  some algorithms such as \gls{CBC mode}. Generally not required to be
  secret, but required to be unpredictable.
  Compare \gls{nonce}, \gls{salt}} }
\newglossaryentry{CTR mode}{
  name=CTR mode, description={Counter mode; a \gls{nonce} combined
  with a counter produces a sequence of inputs to the block cipher;
  the resulting ciphertext blocks are the keystream} }
\newglossaryentry{nonce}{
  name=nonce, description={\emph{N}umber used \emph{once}. Used in
  many cryptographic protocols. Generally does not have to be secret
  or unpredictable, but does have to be unique.
  Compare \gls{initialization vector}, \gls{salt}} }
\newglossaryentry{AEAD mode}{
  name=AEAD mode, description={Class of \gls{block cipher} \glspl{mode
  of operation} that provides authenticated encryption, as well as
  authenticating some unencrypted associated data}}
\newglossaryentry{OCB mode}{
  name=OCB mode, description={Offset codebook mode;
  high-performance \gls{AEAD mode}, unfortunately encumbered by
  patents} }
\newglossaryentry{GCM mode}{
  name=GCM mode, description={Galois counter mode; \gls{AEAD mode}
  combining \gls{CTR mode} with a \gls{Carter-Wegman MAC}}}
\newglossaryentry{message authentication code}{
  name=message authentication code, description={Small piece of
  information used to verify authenticity and integrity of a message.
  Often called a tag}}
\newglossaryentry{one-time MAC}{
  name=one-time MAC, description={\Gls{message authentication code}
  that can only be used securely for a single message. Main benefit is
  increased performance over re-usable \glspl{MAC}}}
\newglossaryentry{Carter-Wegman MAC}{
  name=Carter-Wegman MAC, description={Reusable \gls{message
  authentication code} scheme built from a \gls{one-time MAC}.
  Combines benefits of performance and ease of use}}
\newglossaryentry{GMAC}{
  name=GMAC, description={\Gls{message authentication code}
  part of \gls{GCM mode} used separately}}
\newglossaryentry{salt}{
  name=salt, description={Random data that is added to a cryptographic
  primitive (usually a one-way function such as a cryptographic hash
  function or a key derivation function) Customizes such functions to
  produce different outputs (provided the salt is different). Can be
  used to prevent e.g. dictionary attacks. Typically does not have to
  be secret, but secrecy may improve security properties of the
  system. Compare \gls{nonce}, \gls{initialization vector}} }
\newglossaryentry{public-key algorithm}{
  name=public-key algorithm, description={Algorithm that uses a pair
  of two related but distinct keys. Also known
  as \glspl{asymmetric-key algorithm}. Examples
  include \gls{public-key encryption} and most \gls{key exchange}
  protocols} }
\newglossaryentry{asymmetric-key algorithm}{
  name=asymmetric-key algorithm, description={See \gls{public-key
  algorithm}} }
\newglossaryentry{public-key encryption}{
  name=public-key encryption, description={Encryption using a pair of
  distinct keys for encryption and decryption. Also known as
  asymmetric-key encryption. Contrast with \gls{secret-key
  encryption}} }
\newglossaryentry{asymmetric-key encryption}{
  name=asymmetric-key encryption, description={See \gls{public-key
  encryption}} }
\newglossaryentry{key exchange}{
  name=key exchange, description={The process of exchanging keys
  across an insecure medium using a particular cryptographic protocol.
  Typically designed to be secure against eavesdroppers. Also known
  as key agreement} }
\newglossaryentry{key agreement}{
  name=key agreement, description={See \gls{key exchange}}}
\newglossaryentry{oracle}{
  name=oracle, description={A \enquote{black box} that will perform
  some computation for you}}
\newglossaryentry{encryption oracle}{
  name=encryption oracle, description={An \gls{oracle} that will
  encrypt some data}}
\newglossaryentry{OTR messaging}{
  name=OTR messaging, description={Off-the-record messaging, messaging
  protocol that intends to mimic the properties of a real-live private
  conversation. Piggy-backs onto existing instant messaging
  protocols}}

\makeglossaries
\usepackage{makeidx}
\makeindex

% Localization
\usepackage{polyglossia}
\setdefaultlanguage{english}

% Mathematical symbols
%% \usepackage{savesym}
%% \savesymbol{iint}
%% \savesymbol{iiint}
\usepackage{amsmath}
\usepackage{amssymb}
\newcommand{\xor}{\oplus}
\newcommand{\madd}{\boxplus}

% Advanced command definitions
\usepackage{xparse}

% Figures and subfigures
\usepackage{float}
\usepackage{rotating}
\usepackage{subcaption}
\usepackage{array}
\usepackage{calc}
\usepackage{framed}
\usepackage{wrapfig}
\usepackage{longtable}
\newcolumntype{C}[1]{>{\centering\arraybackslash$}p{#1}<{$}}
\usepackage[export]{adjustbox}

% Fancy figure command
\DeclareDocumentCommand{\illustration}{
   m     % sub-path to illustration, /Illustrations/{this}.pdf, mandatory
   O{.8} % width as fraction of line width, optional
   o     % caption, optional
   o     % label, optional
  }{
  \begin{figure}[ht!]
    \centering
    \includegraphics[width=#2\linewidth]{./Illustrations/#1.pdf}
    \IfValueTF{#3}{\caption{#3}}{}
    \IfValueTF{#4}{\label{#4}}{}
  \end{figure}
}

% Special subsection commands
\definecolor{shadecolor}{HTML}{AABAD1}
\DeclareDocumentCommand{\advanced}{
   o     % extra caption, optional
  }{
  \begin{shaded}
      \begin{wrapfigure}{l}{.2\linewidth}
          \vspace{-8pt}
          \includegraphics[width=\linewidth]{./Illustrations/Propeller/Propeller.pdf}
      \end{wrapfigure}

      \noindent This is an optional, in-depth section. It almost
      certainly won't help you write better software, so feel free to
      skip it. It is only here to satisfy your inner geek's curiosity.

      \IfValueTF{#1}{#1}{}
  \end{shaded}
}

% Chapter markup
\usepackage{tikz, blindtext}
\makechapterstyle{box}{
  \renewcommand*{\printchaptername}{}
  \renewcommand*{\printchapternum}{
    \flushright
    \begin{tikzpicture}
      \draw[fill,color=black] (0,0) rectangle (2cm,2cm);
      \draw[color=white] (1cm,1cm) node { \chapnumfont\thechapter };
    \end{tikzpicture}
  }
  \renewcommand*{\printchaptertitle}[1]{\flushright\chaptitlefont##1}
}
\chapterstyle{box}

% Title page markup
\usepackage{geometry}
\usepackage{titlesec}
\makeatletter
\newlength\drop
\renewcommand{\maketitle}{
  \thispagestyle{empty}
  \begingroup
  \drop = 0.1\textheight
  \vspace*{\baselineskip}
  \vfill
  \hbox{%
    \hspace*{0.2\textwidth}%
    \rule{1pt}{\dimexpr\textheight-28pt\relax}%
    \hspace*{0.05\textwidth}%
    \parbox[b]{0.75\textwidth}{%
      \vbox{%
        \vspace{\drop}
               {\Huge\bfseries\raggedright\@title\par}\vskip2.37\baselineskip
               {\Large\bfseries\@author\par}
               \vspace{0.5\textheight}
      }% end of vbox
    }% end of parbox
  }% end of hbox
  \vfill
  \null
  \endgroup

  \thispagestyle{plain}
  \par
  \noindent
  Copyright 2013-2019, Laurens Van Houtven (lvh)

  \noindent
  This work is available under the Creative Commons
  Attribution-NonCommercial 4.0 International (CC BY-NC 4.0) license.
  You can find the full text of the license
  at \url{https://creativecommons.org/licenses/by-nc/4.0/}.

  \begin{center}
    \includegraphics{./Illustrations/CC/CC-BY-NC.pdf}
  \end{center}

  \noindent
  The following is a human-readable summary of (and not a substitute
  for) the license. You can:

  \begin{itemize}
  \item Share: copy and redistribute the material in any medium or format
  \item Adapt: remix, transform, and build upon the material
  \end{itemize}

  The licensor cannot revoke these freedoms as long as you follow the
  license terms:

  \begin{itemize}
  \item Attribution: you must give appropriate credit, provide a link
  to the license, and indicate if changes were made. You may do so in
  any reasonable manner, but not in any way that suggests the licensor
  endorses you or your use.
  \item NonCommercial: you may not use the material for commercial
  purposes.
  \item No additional restrictions: you may not apply legal terms or
  technological measures that legally restrict others from doing
  anything the license permits.
  \end{itemize}

  You do not have to comply with the license for elements of the
  material in the public domain or where your use is permitted by an
  applicable exception or limitation.

  No warranties are given. The license may not give you all of the
  permissions necessary for your intended use. For example, other
  rights such as publicity, privacy, or moral rights may limit how you
  use the material.

 \clearpage

 \thispagestyle{plain}
 \par
 \vspace*{.3\textheight}{
   \centering
   \emph{Pomidorkowi}
   \par
   \clearpage
 }
}
\makeatother
